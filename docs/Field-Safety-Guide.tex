% Options for packages loaded elsewhere
\PassOptionsToPackage{unicode}{hyperref}
\PassOptionsToPackage{hyphens}{url}
%
\documentclass[
]{book}
\title{Field Operations Safety Manual}
\author{University of California, Office of the President}
\date{2022-04-06}

\usepackage{amsmath,amssymb}
\usepackage{lmodern}
\usepackage{iftex}
\ifPDFTeX
  \usepackage[T1]{fontenc}
  \usepackage[utf8]{inputenc}
  \usepackage{textcomp} % provide euro and other symbols
\else % if luatex or xetex
  \usepackage{unicode-math}
  \defaultfontfeatures{Scale=MatchLowercase}
  \defaultfontfeatures[\rmfamily]{Ligatures=TeX,Scale=1}
\fi
% Use upquote if available, for straight quotes in verbatim environments
\IfFileExists{upquote.sty}{\usepackage{upquote}}{}
\IfFileExists{microtype.sty}{% use microtype if available
  \usepackage[]{microtype}
  \UseMicrotypeSet[protrusion]{basicmath} % disable protrusion for tt fonts
}{}
\makeatletter
\@ifundefined{KOMAClassName}{% if non-KOMA class
  \IfFileExists{parskip.sty}{%
    \usepackage{parskip}
  }{% else
    \setlength{\parindent}{0pt}
    \setlength{\parskip}{6pt plus 2pt minus 1pt}}
}{% if KOMA class
  \KOMAoptions{parskip=half}}
\makeatother
\usepackage{xcolor}
\IfFileExists{xurl.sty}{\usepackage{xurl}}{} % add URL line breaks if available
\IfFileExists{bookmark.sty}{\usepackage{bookmark}}{\usepackage{hyperref}}
\hypersetup{
  pdftitle={Field Operations Safety Manual},
  pdfauthor={University of California, Office of the President},
  hidelinks,
  pdfcreator={LaTeX via pandoc}}
\urlstyle{same} % disable monospaced font for URLs
\usepackage{color}
\usepackage{fancyvrb}
\newcommand{\VerbBar}{|}
\newcommand{\VERB}{\Verb[commandchars=\\\{\}]}
\DefineVerbatimEnvironment{Highlighting}{Verbatim}{commandchars=\\\{\}}
% Add ',fontsize=\small' for more characters per line
\usepackage{framed}
\definecolor{shadecolor}{RGB}{248,248,248}
\newenvironment{Shaded}{\begin{snugshade}}{\end{snugshade}}
\newcommand{\AlertTok}[1]{\textcolor[rgb]{0.94,0.16,0.16}{#1}}
\newcommand{\AnnotationTok}[1]{\textcolor[rgb]{0.56,0.35,0.01}{\textbf{\textit{#1}}}}
\newcommand{\AttributeTok}[1]{\textcolor[rgb]{0.77,0.63,0.00}{#1}}
\newcommand{\BaseNTok}[1]{\textcolor[rgb]{0.00,0.00,0.81}{#1}}
\newcommand{\BuiltInTok}[1]{#1}
\newcommand{\CharTok}[1]{\textcolor[rgb]{0.31,0.60,0.02}{#1}}
\newcommand{\CommentTok}[1]{\textcolor[rgb]{0.56,0.35,0.01}{\textit{#1}}}
\newcommand{\CommentVarTok}[1]{\textcolor[rgb]{0.56,0.35,0.01}{\textbf{\textit{#1}}}}
\newcommand{\ConstantTok}[1]{\textcolor[rgb]{0.00,0.00,0.00}{#1}}
\newcommand{\ControlFlowTok}[1]{\textcolor[rgb]{0.13,0.29,0.53}{\textbf{#1}}}
\newcommand{\DataTypeTok}[1]{\textcolor[rgb]{0.13,0.29,0.53}{#1}}
\newcommand{\DecValTok}[1]{\textcolor[rgb]{0.00,0.00,0.81}{#1}}
\newcommand{\DocumentationTok}[1]{\textcolor[rgb]{0.56,0.35,0.01}{\textbf{\textit{#1}}}}
\newcommand{\ErrorTok}[1]{\textcolor[rgb]{0.64,0.00,0.00}{\textbf{#1}}}
\newcommand{\ExtensionTok}[1]{#1}
\newcommand{\FloatTok}[1]{\textcolor[rgb]{0.00,0.00,0.81}{#1}}
\newcommand{\FunctionTok}[1]{\textcolor[rgb]{0.00,0.00,0.00}{#1}}
\newcommand{\ImportTok}[1]{#1}
\newcommand{\InformationTok}[1]{\textcolor[rgb]{0.56,0.35,0.01}{\textbf{\textit{#1}}}}
\newcommand{\KeywordTok}[1]{\textcolor[rgb]{0.13,0.29,0.53}{\textbf{#1}}}
\newcommand{\NormalTok}[1]{#1}
\newcommand{\OperatorTok}[1]{\textcolor[rgb]{0.81,0.36,0.00}{\textbf{#1}}}
\newcommand{\OtherTok}[1]{\textcolor[rgb]{0.56,0.35,0.01}{#1}}
\newcommand{\PreprocessorTok}[1]{\textcolor[rgb]{0.56,0.35,0.01}{\textit{#1}}}
\newcommand{\RegionMarkerTok}[1]{#1}
\newcommand{\SpecialCharTok}[1]{\textcolor[rgb]{0.00,0.00,0.00}{#1}}
\newcommand{\SpecialStringTok}[1]{\textcolor[rgb]{0.31,0.60,0.02}{#1}}
\newcommand{\StringTok}[1]{\textcolor[rgb]{0.31,0.60,0.02}{#1}}
\newcommand{\VariableTok}[1]{\textcolor[rgb]{0.00,0.00,0.00}{#1}}
\newcommand{\VerbatimStringTok}[1]{\textcolor[rgb]{0.31,0.60,0.02}{#1}}
\newcommand{\WarningTok}[1]{\textcolor[rgb]{0.56,0.35,0.01}{\textbf{\textit{#1}}}}
\usepackage{longtable,booktabs,array}
\usepackage{calc} % for calculating minipage widths
% Correct order of tables after \paragraph or \subparagraph
\usepackage{etoolbox}
\makeatletter
\patchcmd\longtable{\par}{\if@noskipsec\mbox{}\fi\par}{}{}
\makeatother
% Allow footnotes in longtable head/foot
\IfFileExists{footnotehyper.sty}{\usepackage{footnotehyper}}{\usepackage{footnote}}
\makesavenoteenv{longtable}
\usepackage{graphicx}
\makeatletter
\def\maxwidth{\ifdim\Gin@nat@width>\linewidth\linewidth\else\Gin@nat@width\fi}
\def\maxheight{\ifdim\Gin@nat@height>\textheight\textheight\else\Gin@nat@height\fi}
\makeatother
% Scale images if necessary, so that they will not overflow the page
% margins by default, and it is still possible to overwrite the defaults
% using explicit options in \includegraphics[width, height, ...]{}
\setkeys{Gin}{width=\maxwidth,height=\maxheight,keepaspectratio}
% Set default figure placement to htbp
\makeatletter
\def\fps@figure{htbp}
\makeatother
\setlength{\emergencystretch}{3em} % prevent overfull lines
\providecommand{\tightlist}{%
  \setlength{\itemsep}{0pt}\setlength{\parskip}{0pt}}
\setcounter{secnumdepth}{5}
\usepackage{booktabs}
\ifLuaTeX
  \usepackage{selnolig}  % disable illegal ligatures
\fi
\usepackage[]{natbib}
\bibliographystyle{plainnat}

\begin{document}
\maketitle

{
\setcounter{tocdepth}{1}
\tableofcontents
}
\hypertarget{uc-field-operations-manual}{%
\chapter*{UC Field Operations Manual}\label{uc-field-operations-manual}}
\addcontentsline{toc}{chapter}{UC Field Operations Manual}

This manual provides guidelines and resources to lead safe, successful field courses and research trips. The content focuses on risk management issues that are relevant for California-based field courses and research, international trips, research expeditions, and other outdoor excursions. Field sites may include field stations, natural reserves, public lands or parks, wilderness areas, coastline or waterways, or more controlled sites such as construction areas, excavations, or mines. The Field Operations Manual was developed to serve as a reference document and teaching tool as well as to highlight applicable UC policies and State/Federal laws. The manual is organized into key sections on \textbf{planning, training, incident response, best practices for trip leaders,} and appendices on \textbf{common field hazards} and \textbf{local campus resources.}

Integration of field safety planning into routine instruction and training will meet key objectives and regulatory requirements of your Campus or Department's Injury and Illness Prevention Plan (IIPP). The IIPP is a written safety program to protect employees from illnesses and injuries per the California Code of Regulations Title 8, Section 3203, by establishing a safety management framework for identifying and correcting workplace hazards, ensuring employee training and compliance, and communicating information related to safety and health issues. Faculty, staff and students, including student employees and volunteers, are accountable for health and safety rules and following safe work practices, including:

\begin{itemize}
\tightlist
\item
  obtaining appropriate training for designated activities
\item
  using personal protective equipment (PPE) and safety equipment as required and directed,
\item
  reporting unsafe conditions, malfunctioning equipment, and other safety concerns,
\item
  reporting all injuries and incidents, and
\item
  understanding what to do in the event of an emergency.
\end{itemize}

\begin{center}\includegraphics[width=0.75\linewidth]{images/background1} \end{center}

\hypertarget{copyright-notice}{%
\section*{Copyright Notice}\label{copyright-notice}}
\addcontentsline{toc}{section}{Copyright Notice}

Copyright © 2019 The Regents of the University of California. Except as otherwise noted, this document is subject to a Creative Commons Attribution-NonCommercial 4.0 International Public License (CC BY-NC 4.0). More information about this license is found at: creativecommons.org/licenses/by-nc/4.0/legalcode.

Attribution should be given to ``University of California Office of the President -- Environment, Health \& Safety.'' Other than for attribution purposes, use of the University of California's name is prohibited pursuant to Section 92000 of California's Education Code.

\hypertarget{introduction}{%
\chapter*{Introduction}\label{introduction}}
\addcontentsline{toc}{chapter}{Introduction}

Risk, and recognizing the possibility of loss or injury, is integral to experiential learning and is inherent in field environments where we teach and conduct research. A field instructor or researcher must also be an effective risk manager who understands and anticipates risks and acts appropriately to reduce the likelihood of negative consequences. Accidents often result from a combination of challenging conditions, inadequate preparation and poor communication. For this reason, an effective trip leader must incorporate many attributes of leadership including preparation, competency, effective communication, appropriate judgment, self and group awareness, and tolerance for adversity and uncertainty (adapted from the National Outdoor Leadership School Educator Notebook).

\hypertarget{planning}{%
\chapter{Planning}\label{planning}}

\hypertarget{assess-potential-field-hazards}{%
\section{Assess Potential Field Hazards}\label{assess-potential-field-hazards}}

Hazard assessment for field activities may be triggered by various entities, such as via animal protocol review, as part of the research/lab safety program at your campus, or through department procedures. The field hazard assessment tool below provides an overview of resources and hazard mitigation steps for common UC field activities.

\textbf{All} fieldwork warrants a pre-trip discussion regarding foreseen hazards, appropriate precautions, communication options, and emergency procedures. Additional actions are listed below.

\begin{longtable}[]{@{}
  >{\raggedright\arraybackslash}p{(\columnwidth - 2\tabcolsep) * \real{0.50}}
  >{\raggedright\arraybackslash}p{(\columnwidth - 2\tabcolsep) * \real{0.50}}@{}}
\toprule
\begin{minipage}[b]{\linewidth}\raggedright
\textbf{Destination}
\end{minipage} & \begin{minipage}[b]{\linewidth}\raggedright
\end{minipage} \\
\midrule
\endhead
Will you be traveling more than 100
miles from your home campus/office? & \begin{minipage}[t]{\linewidth}\raggedright
\begin{itemize}
\tightlist
\item
  Register with UC Away for travel
  insurance documentation, 24/7
  assistance, and a custom ``Trip Brief''
\end{itemize}
\end{minipage} \\
Will you be traveling internationally? & \begin{minipage}[t]{\linewidth}\raggedright
\begin{itemize}
\tightlist
\item
  Be familiar with the UC International
  Activities Policy, your campus
  International Activities office
  (listed on p.13 of the Policy);
  the resources at ucgo.org
\end{itemize}
\end{minipage} \\
Does your ``Trip Brief'', the CDC, or
State Department recommend
vaccinations or prophylaxis for your
destination? & \begin{minipage}[t]{\linewidth}\raggedright
\begin{itemize}
\tightlist
\item
  Schedule a medical visit at least 6-8
  weeks prior to your trip;
  Occupational Health, Travel Clinic or
  Student Health Clinics available,
  depending on your campus.
\end{itemize}
\end{minipage} \\
Will you be visiting sites with
hazardous terrain, climate, wildfire,
zoonotic risks, poor sanitation, other
environmental hazards, or remote sites
with limited services (e.g more than
30 minutes from emergency medical
services)? & \begin{minipage}[t]{\linewidth}\raggedright
\begin{itemize}
\item
  Complete a Field Safety Plan and
  review with all participants.
\item
  At least one participant should have
  current first aid training and carry
  a first aid kid.
\end{itemize}
\end{minipage} \\
Does your worksite lack reliable phone
service? & \begin{minipage}[t]{\linewidth}\raggedright
\begin{itemize}
\item
  Include check-in procedures in your
\item
  Avoid working alone, when possible
\item
  Carry field radios or satellite
  communication device
\end{itemize}
\end{minipage} \\
Will you be visiting controlled sites
such as construction sites or mines? & \begin{minipage}[t]{\linewidth}\raggedright
\begin{itemize}
\item
  Request PPE and site access
  requirements in advance
\item
  Carry UC identification
\item
  Avoid working alone, when possible
\item
  Check-in with site manager/
  superintendent to understand what
  other hazards are currently present
  on the job-site
\end{itemize}
\end{minipage} \\
Will you be driving to your
destination via UC, rental or personal
vehicles? & \begin{minipage}[t]{\linewidth}\raggedright
\begin{itemize}
\tightlist
\item
  Review UC auto insurance policies for
  students, faculty and staff; complete
  relevant driver safety training as
  required by your campus; consider off
  road/4x4 training if applicable
\end{itemize}
\end{minipage} \\
Will anyone be chartering boats,
planes or using other non-commercial
means of transportation? & \begin{minipage}[t]{\linewidth}\raggedright
\begin{itemize}
\tightlist
\item
  Consult with Risk Services regarding
  appropriate insurance precautions
\end{itemize}
\end{minipage} \\
\bottomrule
\end{longtable}

\begin{longtable}[]{@{}
  >{\raggedright\arraybackslash}p{(\columnwidth - 2\tabcolsep) * \real{0.50}}
  >{\raggedright\arraybackslash}p{(\columnwidth - 2\tabcolsep) * \real{0.50}}@{}}
\toprule
\begin{minipage}[b]{\linewidth}\raggedright
\textbf{Participation}
\end{minipage} & \begin{minipage}[b]{\linewidth}\raggedright
\end{minipage} \\
\midrule
\endhead
Are you responsible for students
registered in a field course? & \begin{minipage}[t]{\linewidth}\raggedright
\begin{itemize}
\tightlist
\item
  Review UC Field Ops Manual Ch.4:
  ``Best Practices for Trip Leaders'' and
  ``Campus Resources''
\end{itemize}
\end{minipage} \\
Will participants be camping or
sleeping in shared dorms, housing,
etc.? & \begin{minipage}[t]{\linewidth}\raggedright
\begin{itemize}
\item
  Consider establishing a ``Student
  Behavior Agreement'' or reviewing a
  ``Code of Conduct''
\item
  Consider establishing a ``Student
  Behavior Agreement'' or reviewing a
  ``Code of Conduct''
\item
  Set the tone for a safe trip by
  discussing expectations and rules
  before the trip
\item
  Carry a participant roster with
  emergency contact information at
  all times
\end{itemize}
\end{minipage} \\
Will volunteers be helping on your
project? & \begin{minipage}[t]{\linewidth}\raggedright
\begin{itemize}
\tightlist
\item
  Registered volunteers formally;
  consult with Risk
\end{itemize}
\end{minipage} \\
Will family members, partners, or
other companions be travelling with
participants? & \begin{minipage}[t]{\linewidth}\raggedright
\begin{itemize}
\tightlist
\item
  Companions should be registered via
  UC Away and may be eligible for UC
  travel benefits
\end{itemize}
\end{minipage} \\
\bottomrule
\end{longtable}

\begin{longtable}[]{@{}
  >{\raggedright\arraybackslash}p{(\columnwidth - 2\tabcolsep) * \real{0.50}}
  >{\raggedright\arraybackslash}p{(\columnwidth - 2\tabcolsep) * \real{0.50}}@{}}
\toprule
\begin{minipage}[b]{\linewidth}\raggedright
\textbf{FIELD ACTIVITIES-Specifics to}
\textbf{integrate into your Field Safety}
\textbf{Plan}\textbar{}
\end{minipage} & \begin{minipage}[b]{\linewidth}\raggedright
\end{minipage} \\
\midrule
\endhead
Working outdoors with temperatures
over 80 degrees F? & \begin{minipage}[t]{\linewidth}\raggedright
\begin{itemize}
\item
  Complete Heat Illness Prevention
  training
\item
  Carry sufficient water, take breaks
  in shade
\item
  Carry shades or tarps if natural is
  unavailable
\item
  Maintain means of communication,
  awareness of worksite location, and
  ability to obtain EMS
\end{itemize}
\end{minipage} \\
Working in dry vegetation/areas with
high fire danger? & \begin{minipage}[t]{\linewidth}\raggedright
\begin{itemize}
\item
  Complete fire extinguisher training
\item
  Carry a fire extinguisher, shovel,
  and bucket of sand in your vehicle
\item
  Consult with your Campus Fire Marshal
  or Fire Prevention Office
\end{itemize}
\end{minipage} \\
Working in cold, possibly wet
conditions? & \begin{minipage}[t]{\linewidth}\raggedright
\begin{itemize}
\item
  Provide all participants a
  recommended gear list including
  waterproof clothing, boots; layers
  for insulation, extra dry socks, tarp
  etc.
\item
  Carry extra blankets or sleeping bag
  in your vehicle for emergencies
\end{itemize}
\end{minipage} \\
Does work involve:

Excavating soil more than 4 feet deep?

Working at heights over 6 feet?

Entering caves, vaults, mines, or
other potential confined spaces?

Handling or transporting hazardous
materials or samples?

Use of powered tools or equipment?

Working in loud noise?

ATVS?

Snowmobiles? & \begin{minipage}[t]{\linewidth}\raggedright
\begin{itemize}
\item
  Contact EH\&S for appropriate hazard
  assessment, training, and PPE
  selection
\item
  Include training requirements and
  precautions in your Field Safety Plan
  or refer to specific procedures,
  JHAs, etc.
\item
  If medical clearance or vaccinations
  are required, schedule your
  appointment with Occupational Health
  at least 6-8 weeks prior to travel
  (e.g for use of respirators, working
  in loud noise, handling bats or
  other hazardous wildlife).
\end{itemize}
\end{minipage} \\
\bottomrule
\end{longtable}

\hypertarget{captioned-figures-and-tables}{%
\section{Captioned figures and tables}\label{captioned-figures-and-tables}}

Figures and tables \emph{with captions} can also be cross-referenced from elsewhere in your book using \texttt{\textbackslash{}@ref(fig:chunk-label)} and \texttt{\textbackslash{}@ref(tab:chunk-label)}, respectively.

Don't miss Table \ref{tab:nice-tab}.

\begin{Shaded}
\begin{Highlighting}[]
\NormalTok{knitr}\SpecialCharTok{::}\FunctionTok{kable}\NormalTok{(}
  \FunctionTok{head}\NormalTok{(pressure, }\DecValTok{10}\NormalTok{), }\AttributeTok{caption =} \StringTok{\textquotesingle{}Here is a nice table!\textquotesingle{}}\NormalTok{,}
  \AttributeTok{booktabs =} \ConstantTok{TRUE}
\NormalTok{)}
\end{Highlighting}
\end{Shaded}

\begin{table}

\caption{\label{tab:nice-tab}Here is a nice table!}
\centering
\begin{tabular}[t]{rr}
\toprule
temperature & pressure\\
\midrule
0 & 0.0002\\
20 & 0.0012\\
40 & 0.0060\\
60 & 0.0300\\
80 & 0.0900\\
\addlinespace
100 & 0.2700\\
120 & 0.7500\\
140 & 1.8500\\
160 & 4.2000\\
180 & 8.8000\\
\bottomrule
\end{tabular}
\end{table}

\hypertarget{parts}{%
\chapter{Parts}\label{parts}}

You can add parts to organize one or more book chapters together. Parts can be inserted at the top of an .Rmd file, before the first-level chapter heading in that same file.

Add a numbered part: \texttt{\#\ (PART)\ Act\ one\ \{-\}} (followed by \texttt{\#\ A\ chapter})

Add an unnumbered part: \texttt{\#\ (PART\textbackslash{}*)\ Act\ one\ \{-\}} (followed by \texttt{\#\ A\ chapter})

Add an appendix as a special kind of un-numbered part: \texttt{\#\ (APPENDIX)\ Other\ stuff\ \{-\}} (followed by \texttt{\#\ A\ chapter}). Chapters in an appendix are prepended with letters instead of numbers.

\hypertarget{footnotes-and-citations}{%
\chapter{Footnotes and citations}\label{footnotes-and-citations}}

\hypertarget{footnotes}{%
\section{Footnotes}\label{footnotes}}

Footnotes are put inside the square brackets after a caret \texttt{\^{}{[}{]}}. Like this one \footnote{This is a footnote.}.

\hypertarget{citations}{%
\section{Citations}\label{citations}}

Reference items in your bibliography file(s) using \texttt{@key}.

For example, we are using the \textbf{bookdown} package \citep{R-bookdown} (check out the last code chunk in index.Rmd to see how this citation key was added) in this sample book, which was built on top of R Markdown and \textbf{knitr} \citep{xie2015} (this citation was added manually in an external file book.bib).
Note that the \texttt{.bib} files need to be listed in the index.Rmd with the YAML \texttt{bibliography} key.

The RStudio Visual Markdown Editor can also make it easier to insert citations: \url{https://rstudio.github.io/visual-markdown-editing/\#/citations}

\hypertarget{blocks}{%
\chapter{Blocks}\label{blocks}}

\hypertarget{equations}{%
\section{Equations}\label{equations}}

Here is an equation.

\begin{equation} 
  f\left(k\right) = \binom{n}{k} p^k\left(1-p\right)^{n-k}
  \label{eq:binom}
\end{equation}

You may refer to using \texttt{\textbackslash{}@ref(eq:binom)}, like see Equation \eqref{eq:binom}.

\hypertarget{theorems-and-proofs}{%
\section{Theorems and proofs}\label{theorems-and-proofs}}

\hypertarget{callout-blocks}{%
\section{Callout blocks}\label{callout-blocks}}

The R Markdown Cookbook provides more help on how to use custom blocks to design your own callouts: \url{https://bookdown.org/yihui/rmarkdown-cookbook/custom-blocks.html}

\hypertarget{sharing-your-book}{%
\chapter{Sharing your book}\label{sharing-your-book}}

\hypertarget{publishing}{%
\section{Publishing}\label{publishing}}

HTML books can be published online, see: \url{https://bookdown.org/yihui/bookdown/publishing.html}

\hypertarget{pages}{%
\section{404 pages}\label{pages}}

By default, users will be directed to a 404 page if they try to access a webpage that cannot be found. If you'd like to customize your 404 page instead of using the default, you may add either a \texttt{\_404.Rmd} or \texttt{\_404.md} file to your project root and use code and/or Markdown syntax.

\hypertarget{metadata-for-sharing}{%
\section{Metadata for sharing}\label{metadata-for-sharing}}

Bookdown HTML books will provide HTML metadata for social sharing on platforms like Twitter, Facebook, and LinkedIn, using information you provide in the \texttt{index.Rmd} YAML. To setup, set the \texttt{url} for your book and the path to your \texttt{cover-image} file. Your book's \texttt{title} and \texttt{description} are also used.

This \texttt{gitbook} uses the same social sharing data across all chapters in your book- all links shared will look the same.

Specify your book's source repository on GitHub using the \texttt{edit} key under the configuration options in the \texttt{\_output.yml} file, which allows users to suggest an edit by linking to a chapter's source file.

Read more about the features of this output format here:

\url{https://pkgs.rstudio.com/bookdown/reference/gitbook.html}

Or use:

\begin{Shaded}
\begin{Highlighting}[]
\NormalTok{?bookdown}\SpecialCharTok{::}\NormalTok{gitbook}
\end{Highlighting}
\end{Shaded}


  \bibliography{book.bib,packages.bib}

\end{document}
